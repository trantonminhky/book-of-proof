\documentclass{exam}
\usepackage{graphicx}
\usepackage[utf8]{inputenc}
\usepackage[english]{babel}
\usepackage{amsmath}
\usepackage{hyperref}
\usepackage{amsthm}
\usepackage{tcolorbox}
\usepackage{amsfonts}
\usepackage{amssymb}
\usepackage{mathrsfs}

\DeclareMathOperator{\lcm}{lcm}

\newbox\eeveebox
\setbox\eeveebox\hbox{
\raisebox{-2.5pt}{\includegraphics[height=2.5ex]{iibui.png}}}
\def\eeveeKawaii{\copy\eeveebox}

\NewTColorBox{proposition}{m}{
  standard jigsaw,
  sharp corners,
  boxrule=0.4pt,
  coltitle=black,
  colframe=black,
  opacityback=0,
  opacitybacktitle=0,
  fonttitle=\normalfont\bfseries\upshape,
  fontupper=\normalfont\itshape,
  title={Proposition #1},
  after title={.},
  attach title to upper={\ },
}

\renewcommand\qedsymbol{$\eeveeKawaii$}

\title{Hammack Exercises - Chapter 7}
\author{FungusDesu}
\date{September 9st 2024}

\begin{document}

\maketitle

\section{Preface}
i dont really have anything to say

\section{Non-Conditional Statements}

\begin{proposition}{7.1$\mathbf{\alpha}$}
    Suppose $x\in\mathbb Z$. Then $x$ is even if and only if $3x + 5$ is odd.
\end{proposition}

\begin{proof}
    We first show that if $x$ is even, then $3x + 5$ is odd. Suppose $x$ is even. Then $x = 2k$ for some $k\in\mathbb Z$. Thus $3x + 5 = 2(3k + 2) + 1$, which is odd by definition of an odd number.

    Conversely, suppose to the contrapositive that $x$ is odd. Then $x = 2l + 1$ for some $l\in\mathbb Z$. Thus $3x +5 = 2(3l + 3)$, which is even by definition of an even number.
\end{proof}

\begin{proposition}{7.2}
    Suppose $x\in\mathbb Z$. Then $x$ is odd if and only if $3x + 6$ is odd.
\end{proposition}

\begin{proof}
    We first show that if $x$ is odd, then $3x + 6$ is odd. Suppose $x$ is odd. Then $x = 2k+1$ for some $k\in\mathbb Z$. Thus $3x + 6 = 2(3k + 4)+1$, which is odd by definition of an odd number.

    Conversely, suppose to the contrapositive that $x$ is even. Then $x = 2l$ for some $l\in\mathbb Z$. Thus $3x + 6 = 2(3l + 3)$, which is even by definition of an even number.
\end{proof}

\begin{proposition}{7.3}
    Given an integer $a$, then $a^3+a^2+a$ is even if and only if $a$ is even.
\end{proposition}

\begin{proof}
    We first show that if $a^3 + a^2 + a$ is even, then $a$ is even. Suppose to the contrapositive that a is odd; we wish to show $a^3+a^2+a$ is odd, or $a(a^2 + a + 1)$ is odd. Since $a$ is odd, we have $a^2 + a +1$ is odd (the square of $a$ is odd, the sum of two odd numbers is an even number, which becomes odd again when added by 1). The product of two odd numbers is odd itself, thus $a(a^2+a+1)$ is odd.

    Conversely, suppose $a$ is even. Regardless of the parity of $a^2 + a + 1$, their product will always be even. Thus the proof is completed.
\end{proof}

\begin{proposition}{7.4}
    Given an integer $a$, then $a^2 + 4a + 5$ is odd if and only if $a$ is even.
\end{proposition}

\begin{proof}
    We first show that if $a^2 + 4a + 5$ is odd, then $a$ is even. Suppose to the contrapositive that $a$ is odd. Notice that $a^2$ is odd and $4a$ is even, thus their sum is an odd number (the sum of two numbers with opposite parity is an odd number). The sum of two odd numbers is an even number, thus $a^2 + 4a + 5$ is even.

    Conversely, suppose $a$ is even. By the same line of reasoning, we can deduce that $a^2 + 4a + 5$ is odd ($a^2$ is even, $4a$ is even, $5$ is odd). The proof is completed.
\end{proof}

\begin{proposition}{7.5}
    An integer $a$ is odd if and only if $a^3$ is odd.
\end{proposition}

\begin{proof}
    We first show that if $a$ is odd, then $a^3$ is odd. Suppose $a$ is odd. Then $a = 2x + 1$ for some $x\in\mathbb Z$. Thus $a^3 =8x^3 + 12x^2 + 6x + 1 = 2(4x^3+6x^2+3x) + 1$. Because $4x^3 + 6x^2 + 3x\in\mathbb Z$, we have $a^3$ is odd by definition of an odd number.

    Conversely, suppose to the contrapositive that $a$ is even. Then $a = 2y$ for some $y\in\mathbb Z$. Thus $a^3 = 8y^3 = 2(4y^3)$. Because $4y^3 \in\mathbb Z$, we have $a^3$ is even by definition of an odd number. The proof is completed.
\end{proof}

\begin{proposition}{7.6}
    Suppose $x, y\in\mathbb R$. Then $x^3 + x^2y=y^2 + xy$ if and only if $y=x^2$ or $y=-x$.
\end{proposition}

\begin{proof}
    $(\impliedby)$ Suppose $y=x^2$ or $y=-x$; the equation is true if $x = y = 0$. We divide into the following two cases for non-zero $x$ and $y$.
    \begin{description}
        \item[Case 1. ] If $y=x^2$, then
        \begin{gather*}
            x^2 = y\\
            x^2y = y^2\\
            x^3 +x^2y = y^2 + x^3\\
            x^3 + x^2y = y^2 +xy.
        \end{gather*}
        \item[Case 2. ] If $y = -x$, then
        \begin{gather*}
            y = -x\\
            y(x^2-y)=-x(x^2-y)\\
            x^2y - y^2 = -x^3 + xy\\
            x^3+x^2y = y^2 + xy.
        \end{gather*} 
    \end{description}
    The cases have shown that $x^3 + x^2y = y^2 + xy$ if $y = x^2$ or $y = -x$.

    $(\implies)$ Suppose $x^3 + x^2y = y^2 + xy$; this implies $(x+y)(x^2-y) = 0$. For this equality to hold, either $x + y = 0$ or $x^2 - y = 0$. Thus $y = -x$ and $y = x^2$. The proof is completed.
\end{proof}

\begin{proposition}{7.7}
    Suppose $x,y\in\mathbb R$. Then $(x+y)^2=x^2+y^2$ if and only if $x=0$ or $y=0$.
\end{proposition}

\begin{proof}
    $(\implies)$ Suppose $(x+y)^2 = x^2 + y^2$ for some $x,y\in\mathbb R$. Then $x^2+2xy +y^2 = x^2 + y^2$ implies $xy = 0$. The equation holds if and only if $x = 0$ or $y = 0$.

    $(\impliedby)$ Suppose $x = 0$. Then $(x+y)^2 = x^2 + y^2$ implies $y^2 = y^2$. The same line of reasoning applies when $y=0$. Thus the proof is completed.
\end{proof}

\begin{proposition}{7.8}
    Suppose $a, b\in\mathbb Z$. Prove that $a\equiv b\pmod{10}$ if and only if $a\equiv b\pmod2$ and $a\equiv b\pmod5$.
\end{proposition}

\begin{proof}
    $(\implies)$ Suppose $a\equiv b\pmod{10}$ for some $a, b\in\mathbb Z$. Then $10\mid (a-b)$. Thus there exists $x$ such that
    $a-b=10x$; because $10x = 2\cdot5x = 5\cdot2x$, we have $2\mid(a-b)$ and $5\mid(a-b)$ consequently. Therefore $a\equiv b\pmod2$ and $a\equiv b\pmod5$.

    $(\impliedby)$ Suppose $a\equiv b\pmod 2$ and $a\equiv b\pmod 5$. Then there exists $m,n$ such that 
    \begin{gather}
        (a-b) = 2m = 5n.
    \end{gather}
    Thus $5 \mid 2m$, which implies $5\mid m$ by Proposition 4.8. As such, there exists $x\in\mathbb Z$ such that $m = 5x$. Substituting it into $(1)$ yields $(a-b)=10x$. We have shown that $10\mid(a-b)$, and consequently $a\equiv b\pmod{10}$. The proof is completed.
\end{proof}

\begin{proposition}{7.9}
    Suppose $a\in\mathbb Z$. Prove that $14\mid a$ if and only if $7\mid a$ and $2\mid a$.
\end{proposition}

\begin{proof}
    $(\implies)$ Suppose $14\mid a$. Then $a = 14x$ for some $x\in\mathbb Z$. Note that $a = 14x = 7\cdot(2x) = 2\cdot(7x)$; thus $7\mid a$ and $2\mid a$.

    $(\impliedby)$ Suppose $7\mid a$ and $2\mid a$. Then there exists $m, n\in\mathbb Z$ such that
    \begin{align}
        a = 7m = 2n.
    \end{align}
    Thus $7\mid 2n$ implies $2n = 7x$ for some $x\in\mathbb Z$. The left hand side is an even number, so the right hand side must be as well. Thus $x$ is a multiple of 2, so $\frac{x}2\in\mathbb Z$. Let $k = \frac{x}2$; we have $n = 7k$. Substituting this into $(2)$, we get $a = 14k$. Thus $14 \mid a$. The proof is completed.
\end{proof}

\begin{proposition}{7.10}
    If $a\in\mathbb Z$, then $a^3\equiv a\pmod3$.
\end{proposition}

\begin{proof}
    Consider the expression $a^3 - a$ for some integer $a$. We can rewrite it into $(a-1)a(a+1)$, which is the product of three consecutive integers. Note that no matter what three numbers we pick, there will always be a multiple of 3. Thus their product will also be, and we are done.
\end{proof}

\begin{proposition}{7.11}
    Suppose $a, b\in\mathbb Z$. Prove that $(a-3)b^2$ is even if and only if $a$ is odd or $b$ is even.
\end{proposition}

\begin{proof}
    $(\implies)$ Suppose to the contrapositive that $a$ is even and $b$ is odd. Then there exist $m,n\in\mathbb Z$ such that $a = 2m$ and $b = 2n + 1$. Thus $(a-3)b^2 = (2m - 3)(2n+1)^2 = 2(4mn^2+4mn+m-6n^2-6n-2) + 1$, which is odd.

    $(\impliedby)$ Suppose that $a$ is odd or $b$ is even. We divide into the following two cases:
    \begin{description}
        \item[Case 1. ] If $a$ is odd, then $a = 2m+1$ for some $m\in\mathbb Z$. Thus $(a-3)b^2=(2m-2)b^2=2(m-1)b^2$, which is even.
        \item[Case 2. ] If $b$ is even, then $b=2n$ for some $n\in\mathbb Z$. Thus $(a-3)b^2 = 4n^2(a-3)$, which is even.
    \end{description}
    The proof is completed.
\end{proof}

\begin{proposition}{7.12}
    There exists a positive real number $x$ for which $x^2<\sqrt x$.
\end{proposition}

\begin{proof}
    Consider $x = \frac14$. Note that $x^2 = \frac1{16}<\sqrt x=\frac12$. Thus $x=\frac14$ is a positive real number that satisfies $x^2 < \sqrt x$.
\end{proof}

\begin{proposition}{7.13}
    Suppose $a, b\in\mathbb Z$. If $a + b$ is odd, then $a^2 + b^2$ is odd.
\end{proposition}

\begin{proof}
    Suppose $(a+b)$ is odd. Then $(a+b)^2 = a^2 + b^2 + 2ab$ is also odd. Note that a sum of two integers is odd if and only if the integers have opposite parity (Reason: Let $m, n$ be any integer. Then $2m + 2n = 2(m+n)$ is even, $2m + 1 + 2n + 1 = 2(m+n+1)$ is even, but $2m + 2n + 1 = 2(m+n) + 1$ is odd. The converse can be proven by reversing this line of reasoning). $2ab$ is even, thus $a^2 + b^2$ must be odd, and we are done.
\end{proof}

\begin{proposition}{7.14}
    Suppose $a\in\mathbb Z$. Then $a^2\mid a$ if and only if $a\in\{-1,0,1\}$.
\end{proposition}

\begin{proof}
    $(\implies)$ Suppose $a^2 \mid a$. Then there exists $k$ such that $a = a^2k$. We divide into two cases as follows:
    \begin{description}
        \item[Case 1. ] If $a = 0$, then the equation $a=a^2k$ is true.
        \item[Case 2. ] If $a\neq 0$, then $a=a^2k$ implies $1=ak$. The equality holds if and only if $a = 1$ and $k = 1$, or $a = -1$ and $k = -1$.
    \end{description}
    The cases have shown that if $a^2 \mid a$, then $a \in\{-1,0,1\}$.

    $(\impliedby)$ Suppose $a\in A=\{-1, 0, 1\}$. We can easily see that all elements of $A$ all satisfy $a^2 \mid a$ ($1\mid -1,0\mid0, 1\mid 1$). The proof is complete.
\end{proof}

\begin{proposition}{7.15}
    Suppose $a, b\in\mathbb Z$. Prove that $a+b$ is even if and only if $a$ and $b$ have the same parity.
\end{proposition}

\begin{proof}
    $(\implies)$ Suppose to the contrapositive that $a$ and $b$ have the opposite parity. Without loss of generality, suppose $a$ is odd and $b$ and even; thus $a = 2m + 1$ and $b = 2n$ for some $m,n\in\mathbb Z$. Then $a + b = 2m + 1 + 2n = 2(m+n)+1$, which is odd.

    $(\impliedby)$ Suppose $a$ and $b$ have the same parity. If $a$ and $b$ are both odd, then there exist $m,n\in\mathbb Z$ such that $a  =2m + 1$ and $b = 2n + 1$; thus $a + b = 2(m+n+1)$, which is even. If $a$ and $b$ are both even, then there exist $m,n\in\mathbb Z$ such that $a=2m$ and $b=2n$; thus $a + b = 2(m+n)$, which is even. The proof is thus completed.
\end{proof}

\begin{proposition}{7.16}
    Suppose $a,b\in\mathbb Z$. If $ab$ is odd, then $a^2 + b^2$ is even.
\end{proposition}

\begin{proof}
    Note that if $ab$ is odd, then $a$ and $b$ must also be odd themselves (Suppose to the contrapositive that $a$ or $b$ is even; then $ab$ must also be even because there is a multiple of two due to $a$ or $b$ being even.). Therefore, there exist $m,n\in\mathbb Z$ such that $a = 2m +1$ and $b=2n+1$. Thus
    \begin{align*}
        a^2+b^2=(2m+1)^2+(2n+1)^2=2(2m^2+2n^2+2m+2n+1).
    \end{align*}
    The proof is complete.
\end{proof}

\begin{proposition}{7.17}
    There is a prime number between 90 and 100.
\end{proposition}

\begin{proof}
    Observe 97.
\end{proof}

\begin{proposition}{7.18}
    There is a set $X$ for which $\mathbb N\in X$ and $\mathbb N\subseteq X$.
\end{proposition}

\begin{proof}
    Observe $X = \mathbb N \cup \{\mathbb N\}$.
\end{proof}

\begin{proposition}{7.19}
    If $n\in\mathbb N$, then $2^0+2^1+2^2+2^3+2^4+\dots+2^n=2^{n+1}-1$.
\end{proposition}

\begin{proof}
    Consider a geometric progression with common ratio $q = 2$. The sum of the first $n+1$ terms is
    \begin{align*}
        2^0 + 2^1 + 2^2 +\dots+2^n = \frac{2^{n+1} - 1}{2-1} = 2^{n+1}-1.
    \end{align*}
\end{proof}

\begin{proposition}{7.20}
    There exists an $n\in\mathbb N$ for which $11\mid(2^n-1)$
\end{proposition}

\begin{proof}
    Consider $n = 10$. Note that $2^10-1 = 1023 = 11\cdot93$. Thus $n=10$ is a possible value of $n$ for which $11\mid(2^n-1)$.
\end{proof}

\begin{proposition}{7.21}
    Every real solution of $x^3+x+3=0$ is irrational.
\end{proposition}

\begin{proof}
    Suppose to the contrary that there exists a rational solution of $x^3 + x +3 = 0$. Let that solution be $x_0=\frac{p}{q}$ for some $p,q\in\mathbb Z$. Let this fraction be irreducible; that is, $\gcd(p, q) = 1$. Substituting this into the original equation, we get:
    \begin{align*}
        \frac{p^3}{q^3} + \frac p q + 3 = 0\\
        \frac{p^3+pq^2}{q^3}+3=0\\
        \frac{p(p^2+q^2)}{q^3}=-3\\
        p^3+pq^2+3q^3=0
    \end{align*}
    We divide into three cases as follows depending on the parity of $p$ and $q$. The case where $p$ and $q$ are both even will not be considered since it contradicts the fact that $p$ and $q$ are coprime.
    \begin{description}
        \item[Case 1. ] Consider odd $p$ and $q$. Note that $p^3$ is odd, $pq^2$ is odd and $3q^3$ is odd. The sum of three odd numbers is odd, which contradicts with the fact that $p^3 + pq^2 + 3q^3 = 0$.
        \item[Case 2. ] Consider odd $p$ and even $q$. Note that $p^3$ is odd, $pq^2$ is even and $3q^3$ is even. The sum of two even numbers and an odd number is odd, contradicting with the fact that $p^3 + pq^2 + 3q^3 = 0$.
        \item[Case 3. ] Consider even $p$ and odd $q$. Note that $p^3$ is even, $pq^2$ is even and $3q^3$ is odd. Similar to Case 2, this is a contradiction.
    \end{description}
    In every possible case, we come into a contradiction. The proof is thus completed.
\end{proof}

\begin{proposition}{7.22}
    If $n\in\mathbb Z$, then $4\mid n^2$ or $4\mid(n^2-1)$.
\end{proposition}

\begin{proof}
    This is proven in Proposition 5.28. The proof is thus completed.
\end{proof}

\begin{proposition}{7.23}
    Suppose $a, b$ and $c$ are integers. If $a\mid b$ and $a\mid(b^2-c)$, then $a\mid c$.
\end{proposition}

\begin{proof}
    Suppose $a\mid b$ and $a\mid (b^2-c)$; then there exists $k\in\mathbb Z$ such that $b = ak$. Thus $b^2 = a(ak^2)$, which implies $a\mid b^2$. Therefore $a\mid b^2$ and $a\mid (b^2-c)$ imply $a\mid c$ by Proposition 4.6. The proof is completed.
\end{proof}

\begin{proposition}{7.24}
    If $a\in\mathbb Z$, then $4\nmid (a^2 - 3)$
\end{proposition}

\begin{proof}
    Proposition 5.28 tells that indeed $4\nmid (a^2 - 3)$.
\end{proof}

\begin{proposition}{7.25}
    If $p>1$ is an integer and $n\nmid p$ for each integer $n$ for which $2\le n\le\sqrt p$, then $p$ is prime.
\end{proposition}

\begin{proof}
    Suppose to the contrapositive that $p$ is not prime, where $p \in\mathbb Z$ and $p > 1$. Then $p=ab$ for some integer $a, b$. Observe that at least one of $a$ or $b$ must be less than or equal $\sqrt{p}$. Thus $p$ must have at least one divisor $n$ such that $2\le n\le\sqrt p$, and we are done.
\end{proof}

\begin{proposition}{7.26}
    The product of any $n$ consecutive positive integers is divisible by $n!$.
\end{proposition}

\begin{proof}
    Let $k$ be the starting number in a sequence of $n$ consecutive positive integers. The product of every number from that sequence is $k(k+1)(k+2)\cdots(k+n-1)$. Thus we wish to prove the following:
    \begin{align*}
        \frac{k(k+1)(k+2)\cdots(k+n-1)}{n!}\in\mathbb Z.
    \end{align*}
    Note that the fraction can be reduced into the following:
    \begin{align*}
        \frac{k(k+1)(k+2)\cdots(k+n-1)}{n!} = \frac{(k+n-1)!}{n!(k-1)!} = \binom{k+n-1}{n}.
    \end{align*}
    By the definition of $\binom{a}{b}$ for some natural $a, b$, we have $\binom{k+n-1}{n}$ is always a natural number. The proof is thus completed.
\end{proof}

\begin{proposition}{7.27}
    Suppose $a,b\in\mathbb Z$. If $a^2+b^2$ is a perfect square, then $a$ and $b$ are not both odd.
\end{proposition}

\begin{proof}
    Suppose to the contrapositive that both $a$ and $b$ are odd. Then $a = 2m+1$ and $b = 2n + 1$ for some $m,n\in\mathbb Z$. Then $a^2 + b^2 = 4(m^2+m+n^2+n) + 2$. Suppose for the sake of contradiction that this is a perfect square. Then there exists $k\in\mathbb Z$ such that $4(m^2 + m +n^2 + n) = k^2 - 2$. The left hand side is a multiple of 4, so the right hand must also be. But only $4\mid k^2$ or $4\mid (k^2-1)$ is true (proven in Proposition 5.28), thus the right hand side cannot be a multiple of 4. This is a contradiction, therefore $4(m^2+m+n^2+n)+2$ must not be a perfect square. The proof is completed.
\end{proof}

\begin{proposition}{7.28}
    If $a,b\in\mathbb N$, there exist unique integers $q,r$ for which $a=bq+r$, and $0\le r < b$.
\end{proposition}

\begin{proof}
    Form the set $A$ for which
    \begin{align*}
        A = \{a - bq: q\in\mathbb Z, a-bq \ge 0\}\subseteq\mathbb N_0.
    \end{align*}
    Let $r$ be the smallest element of $A$. Since $r\in A$, we have $r = a - bq$ implies $a = bq + r$. We know that $r\ge0$ because $r\in A\subseteq \mathbb N_0$. Additionally, it must be true that $r < b$. Suppose to the contrary that $r \ge b$, then the non-negative number $r-b = a-bq -b = a-b(q+1)$ would be an element of $A$ and smaller than $r$. This contradicts the fact that $r$ is the smallest element of $A$, thus $r<b$. We have established the existence of $q,r\in\mathbb Z$ such that $a = bq + r$ and $0\le r<b$.

    To show there exist unique pair of integers $(q, r)$ that satisfy such property, we assume there is a second pair that also does. Let $(q',r')$ be such pair of integers. Thus $a=bq'+r'=bq+r$. This implies $b(q'-q)=r-r'$, which means $r-r'$ is a multiple of $b$. Note that $0 \le r \le b-1$ and $-b+1 \le -r' \le 0$, thus $-b + 1\le r - r' \le b - 1$, implying the only multiple of $b$ in the range of $r-r'$ is 0. Thus $r-r'=0$ implies $r = r'$, and consequently $q' = q$. The uniqueness of $q, r$ has been established, the proof is thus completed.
\end{proof}

\begin{proposition}{7.29}
    If $a\mid bc$ and $\gcd(a,b)=1$, then $a\mid c$.
\end{proposition}

\begin{proof}
    Suppose $a\mid bc$ and $\gcd(a,b)=1$. By Proposition 7.1, then there exist $m, n\in\mathbb Z$ such that $\gcd(a,b) = am + bn = 1$. Note that $amc + bnc = c$. We have $amc$ is a multiple of $a$; because $a\mid bc$, we can see $bc$ is a multiple of $a$, and thus so is $bnc$. The sum of two multiples of $a$ is itself a multiple of $a$, thus $a \mid c$.
\end{proof}

\begin{proposition}{7.30}
    Suppose $a,b,p\in\mathbb Z$ and $p$ is prime. Prove that if $p\mid ab$ then $p\mid a$ or $p\mid b$.
\end{proposition}

\begin{proof}
    Suppose $p \mid ab$. We divide into two cases as follows, depending on the divisibility of $p$ on $a$:
    \begin{description}
        \item[Case 1. ] If $p \mid a$, then we are done.
        \item[Case 2. ] If $p\nmid a$, then that implies $\gcd(a, p) = 1$. This is because $\gcd(a, p) \mid p$, but because $p$ is prime, $\gcd(a, p)$ can only evaluate to $1$ or $p$; the latter happens if and only if $p \mid a$. Since $p \mid ab$ and $\gcd(a, p) = 1$, by Proposition 7.29, we have $p \mid b$.
    \end{description}
    This completes the proof.
\end{proof}

\begin{proposition}{7.31}
    If $n\in\mathbb Z$, then $\gcd(n, n+1) = 1$.
\end{proposition}

\begin{proof}
    By Proposition 5.29.1, we have $\gcd(n, n+1) = \gcd(n+1-n, n) = \gcd(1, n)$, which evaluates to 1.
\end{proof}

\begin{proposition}{7.32}
    If $n\in\mathbb Z$, then $\gcd(n,n+2)\in\{1,2\}$.
\end{proposition}

\begin{proof}
    By Proposition 5.29.1, we have $\gcd(n, n+2) = \gcd(n+2-n, n) = \gcd(n, 2)$. If $n$ is even, then $\gcd(n, 2)$ evaluates to 2; otherwise, $\gcd(n, 2)$ evaluates to 1. Thus $\gcd(n, n+2)\in\{1,2\}$, as desired.
\end{proof}

\begin{proposition}{7.33}
    If $n\in\mathbb Z$, then $\gcd(2n + 1, 4n^2 + 1)=1$.
\end{proposition}

\begin{proof}
    By Proposition 5.29.1, for any integer $n$, we have:
    \begin{align*}
        \gcd(2n+1,4n^2+1)&=\gcd(4n^2+1-4n^2-2n,2n+1)=\gcd(1-2n,2n+1)\\
        &=\gcd(1-2n + 2n + 1, 2n + 1)=\gcd(2, 2n + 1).
    \end{align*}
    Note that $2n + 1$ is odd. Thus $\gcd(2, 2n + 1) = 1$, and we are done.
\end{proof}

\begin{proposition}{7.34}
    If $\gcd(a, c) = \gcd(b, c) = 1$, then $\gcd(ab, c) = 1$.
\end{proposition}

\begin{proof}
    Suppose to the contrary that $\gcd(ab, c) \neq 1$. Let $p$ be a prime number such that $p \mid \gcd(ab, c)$. It follows that $p\mid ab$ and $p\mid c$. Thus $p\mid a$ and $p\mid c$, or $p\mid b$ and $p\mid c$ by Proposition 7.30. But this contradicts the fact that $\gcd(a, c) = \gcd(b, c) = 1$. Thus we have a contradiction.
\end{proof}

\begin{proposition}{7.35}
    Suppose $a, b\in\mathbb N$. Then $a = \gcd(a, b)$ if and only if $a\mid b$.
\end{proposition}

\begin{proof}
    $(\implies)$ Suppose $a = \gcd(a, b)$. Thus there exists $x\in\mathbb Z$ such that $b = \gcd(a, b)x = ax$. Therefore $a \mid b$.

    $(\impliedby)$ Suppose $a \mid b$. Since $a$ divides $b$, every divisor of $a$ will also divide $b$, where $a$ itself is the largest among them. Thus $\gcd(a, b) = a$. This completes the proof.
\end{proof}

\begin{proposition}{7.36}
    Suppose $a,b\in\mathbb N$. Then $a=\lcm(a, b)$ if and only if $b \mid a$
\end{proposition}

\begin{proof}
    $(\implies)$ Suppose $a = \lcm(a,b)$. Thus there exists $x\in\mathbb Z$ such that $\lcm(a, b) = a = bx$. Therefore $b \mid a$.
    $(\impliedby)$ Suppose $b\mid a$. Since $b$ divides $a$, every multiple of $a$ will also be divided by $b$, where $a$ is the lowest among them. Thus $\lcm(a, b) = a$. This completes the proof.
\end{proof}

\end{document}